%!TEX encoding = UTF-8 Unicode
%!TEX program = pdflatex

% A4, dwustronna, rozdziały rozpoczynają się po prawej, czcionka 11pt dla treści
\documentclass[12pt, a4paper, twoside, openright]{book}
% A4, marginesy wg wytycznych, dwustronnie
\usepackage[a4paper, top=25mm, bottom=25mm, inner=30mm, outer=20mm, twoside]{geometry}
\linespread{1.15}
% Kodowanie wyjściowe czcionki
\usepackage[T1]{fontenc}
% Kodowanie plików źródłowych
\usepackage[utf8]{inputenc}
% Załadowanie czcionek Helvetica (odpowiednik Ariala)
\usepackage[scaled]{helvet}
% Załadowanie czcionek latin modern (trochę lepsze od domyślnych computer modern)
\usepackage{lmodern}
% Zgodność czcionek (np textbullet w itemize)
\usepackage{textcomp}
% Język
\usepackage[english,polish]{babel}
% Grafiki w trybie pdftex
\usepackage[pdftex]{graphicx}
% Wcięcie pierwszego akapitu każdej sekcji
\usepackage{indentfirst}
% Wzory matematyczne
\usepackage{amsmath}
\usepackage{gensymb}
% Ustawianie obrazków 'here' ({float}[H]; musi być przed hyperref)
\usepackage{float}
% Podpisy pod obrazkami etc - czcionka 9pt, pogrubiona część 'rysunek x.y:'
\usepackage[font=footnotesize,format=hang,labelfont=bf]{caption}
% Środowisko subfigure ('podobrazki')
\usepackage[font=footnotesize,format=hang,labelfont=bf]{subcaption}
% Automatyczne linki w PDFie (np ze spisu treści do rozdziału) bez przypisów - popsute w pakiecie
\usepackage[hyperfootnotes=false,unicode,pdfusetitle]{hyperref}
% Zaawansowane przypisy - na dole (nie pod tekstem), z możliwymi przypisami w tytułach sekcji, numeracja globalna
\usepackage{chngcntr}
\usepackage[bottom,stable]{footmisc}
% Zaawansowane stopki (ustawienia niżej)
\usepackage{fancyhdr}
% Lepsze tabelki (m.in. m{szerokość})
\usepackage{array}
% Chwilowa zmiana marginesów (w szczególności wyśrodkowanie dużej tabeli)
\usepackage{chngpage}
% Kody źródłowe
\usepackage{listings}
% Algorytmy (pseudokod)
\usepackage[algochapter,boxed,vlined,linesnumbered]{algorithm2e}
% Kolory kodu
\usepackage{xcolor}
% Linki w bibliografii
\usepackage{url}
% Przecinki zamiast kropek w liczbach
\usepackage{icomma}
% Strona tytułowa + oświadczenie
\usepackage{pweiti_titlepage/pweiti_titlepage}


% Skroty
\usepackage{longtable}
\usepackage[acronym]{glossaries}
%% to add tex graph
\usepackage{tikz}
\usetikzlibrary{arrows,calc,decorations.markings,math,arrows.meta}
\usepackage{pgfplots}
\pgfplotsset{compat=newest}

%% Czcionki
% Ustawienie domyślnej czcionki bezszeryfowej na latin modern sans serif
\renewcommand{\sfdefault}{lmss}
% Ustawienie domyślnej czcionki szeryfowej na latin modern roman
\renewcommand{\rmdefault}{lmr}
%\renewcommand{\familydefault}{\sfdefault}
\renewcommand{\familydefault}{\rmdefault}

%%% Typografia
% Wdowy, sieroty etc
\clubpenalty=10000
\widowpenalty=10000
\brokenpenalty=10000

%%% Przypisy i stopki
% Ustawienie przypisów - numeracja liczbami, globalna
\renewcommand{\thefootnote}{\arabic{footnote}}
\counterwithout{footnote}{chapter}
% Ustawienie stopek i nagłówków - brak nagłówków, stopki z numerami po zewnętrznej
\fancyhf{}
\renewcommand{\headrulewidth}{0pt}
\fancyfoot[RO,LE]{\thepage}
\pagestyle{fancyplain}

%%% Spis treści i podsekcje
\setcounter{tocdepth}{2}
\setcounter{secnumdepth}{2}

%%% Listy
\renewcommand{\labelitemi}{$\bullet$}
\renewcommand{\labelitemii}{--}
\renewcommand{\labelitemiii}{$\circ$}

%%% Tablice/tabele
\captionsetup[table]{name=Tabela}

%%% Kody źródłowe
\lstset{
	language=C++,
	basicstyle=\footnotesize\ttfamily,
	keywordstyle=\bfseries\color{green!40!black},
	commentstyle=\itshape\color{purple!40!black},
	directivestyle={\color{blue}},
	% identifierstyle=\color{blue},
	stringstyle=\color{orange},
	emphstyle=\color{blue},
	showstringspaces=true,
	breaklines=true,
	tabsize=4,
	emph={int,char,double,float,unsigned,uint16_t,\*,\&},
	xleftmargin=\parindent,
	captionpos=b,
	numbers=left,
	numbersep=5pt,
	numberstyle=\tiny\color{gray},
	% texcl=true
	literate=
	{=}{{\textcolor{blue}{$=$}}}1
	{*}{{\textcolor{blue}{$*$}}}1
	{\&}{{\textcolor{blue}{$\&$}}}1
	{::}{{\textcolor{blue}{$::$}}}1
}
% Algorytmy
\renewcommand{\algorithmcfname}{Algorytm}%
\SetKwInput{KwIn}{Wejście}%
\SetKwInput{KwOut}{Wyjście}%
\SetAlgoSkip{bigskip}


%%% Strona tytułowa
\author{Karolina Borkowska}
\albumnumber{270800}
\title{Generacja trajektorii w systemie IRPOS}
\supervisor{dr~inż.~Tomasz~Winiarski}
\institute{Instytut~Automatyki i~Informatyki Stosowanej}
\fieldofstudy{Automatyka i Robotyka}
\specialization{}

%%% Bibliografia
\bibliographystyle{plunsrt}

%\renewcommand{\glsclearpage}{}
%\loadglsentries{glossaries}
%\makeglossaries

%%% Treść
\begin{document}
\pagenumbering{Alph}
\maketitle
% TeX encoding = utf8
% !TeX spellcheck = pl_PL

\newpage
\thispagestyle{empty}
\fontsize{12}{12}\selectfont
\vspace*{2\baselineskip}
\begin{center}
	{\large\bfseries Streszczenie}\par\bigskip
\end{center}

\noindent{\bf Tytuł}: {Generacja trajektorii w systemie IRPOS}\par
\vspace*{0.5\baselineskip}

{\itshape
Celem pracy inżynierskiej była modyfikacja systemu IRPOS, umożliwiająca sprawdzenie, czy w~trakcie osiągania przez robota IRp6 punktów zadanej drogi nie zostaną naruszone jego ograniczenia fizyczne i~dynamiczne. A priori przeanalizowana zostaje trajektoria, którą poruszać się ma robot, by osiągnąć zadane położenie.
Zmodernizowany został komponent OROCOS, odpowiedzialny za odbieranie i~wstępne sprawdzenie zadania robota.
Dodatkowo nowa funkcjonalność wstrzymuje dalsze przetwarzanie rozkazu i~związany z~nim ruch. Jest to podejście zgodne z~wcześniej zaimplementowaną koncepcją reakcji na zadanie punktu końcowego niemożliwego do osiągnięcia dla danego stawu.
}

\vspace*{0.5\baselineskip}
\noindent{\bf Słowa kluczowe}: {\itshape robot IRp6, manipulator robotyczny, system IRPOS, planowanie trajektorii manipulatora, OROCOS, Open Robot Control Software.
	}

\par
\vspace{0\baselineskip}
\begin{center}
	{\large\bfseries Abstract}\par\bigskip
\end{center}

\noindent{\bf Title}: {\itshape Generation of robot trajectory in IRPOS system}\par
\vspace*{0.5\baselineskip}

{\itshape
	The aim of this thesis was modificating IRPOS system, that enables checking whether, during IRp6 robot's movement in the direction of setpoints, the physical and dynamic limitations of the robot are not breached. A priori a possible trajectory is analysed. OROCOS component, responsible for receiving and initial movement task test, was modernised.  Moreover, the new functionality blocks further setpoint message processing, thus halting robot's movement. This approach is in accordance with the previously implemented reaction to a setpoint that is impossible to reach by a given joint.
}

\vspace*{0.5\baselineskip}
\noindent{\bf Keywords}: {\itshape IRp6 robot, manipulator robot, IRPOS system, manipulator trajectory planning, OROCOS, Open Robot Control Software .}%, monocular odometry.}

\newpage
\thispagestyle{empty}
\mbox{}
\makeauthorshipstatement
\cleardoublepage
\pagenumbering{arabic}
\tableofcontents{}

% TeX encoding = utf8
% TeX spellcheck = pl_PL 

\chapter{Wstęp}
\label{chapter:wstep}
%Rozdział ten pełni funkcję wstępu do niniejszej pracy. Przybliża on jej tematykę
%i cel oraz pokazuje układ rozdziałów. Przedstawia także motywację,
%dzięki której zagadnienie to zostało zbadane.
	
	
		\section{Wprowadzenie}
		\label{Wprowadzenie}
		%krótki opis zagadnienia
		Mechanizmy imitujące ludzi i~zwierzęta wydają się być częścią wspólnej kulturowej świadomości od początku istnienia cywilizacji. Mitologie Egipcjan, Greków, Hindusów czy Izraelitów wspominają o~urządzeniach, które współczesny czytelnik mógłby nazwać robotami. Największe umysły w historii aspirowały do stworzenia pierwszego autonomicznego pracownika. Wśród pierwszych konstruktorów można znaleźć Leonardo da Vinci'ego, czy Archytas'a z~Tarentu. To, co dla naszych poprzedników pozostawało jedynie fikcją, stało się nieodzowną częścią przemysłu w~XX wieku. Już pierwsze roboty firmy Unimation cechowały się dokładnością znacznie przewyższająca możliwości ludzkie, dodatkowo zapewniając szybszą produkcję oraz większe bezpieczeństwo pracowników. 
		
		Jednym z~najczęściej spotkanych typów robotów przemysłowych jest manipulator. W~swej budowie przypominają one górne kończyny ludzkie. Zakończone są chwytakami, które choć z~zasady mają być odwzorowaniem ludzkiej dłoni, są raczej jej dużym uproszczeniem lub wręcz modyfikacją wprowadzoną do wykonania konkretnego zadania. Człowiek intuicyjnie wybiera rozwiązanie przypominające wytwory natury, przez co manipulatory znajdują szerokie zastosowanie w~najróżniejszych gałęziach przemysłu.  Sprzęt taki jak kamery, taśmy produkcyjne lub tor jezdny, to podstawowe rozszerzenia, jakie dodawane są do systemów robotycznych, w~celu automatyzacji kolejnych etapów produkcji. Od fabryk samochodów, po te skupiające się na wyrobach spożywczych, manipulatory są integralną częścią nowoczesnej cywilizacji.

		
		\section{Motywacja}
		\label{Motywacja}
		System IrpOS zapewnia prosty interfejs do pracy z~robotami IRp-6 znajdującymi się w~laboratorium 012 Wydziału Elektroniki i~Technik Informacyjnych. Za jego pomocą powstają prace dyplomowe, systemy prototypowe oraz prowadzone są zajęcia dydaktyczne. Jednakże istnieją obszary w~których można rozwinąć to oprogramowanie w~celu ułatwienia sterowania robotami. Pierwotnie system w~żaden sposób nie analizował zadanej trajektorii, po czym dodano sprawdzanie czy poszczególne punkty do osiągnięcia znajdują się w~przestrzeni roboczej sprzętu. W~obecnej postaci możliwe jest przekroczenie ograniczeń kinematycznych i~dynamicznych robota, poprzez zadanie zbyt krótkiego czasu na osiągnięcie kolejnych pozycji. Nowa funkcjonalność mogłaby przyśpieszyć tworzenie programów i~wykrywanie błędów, efektywnie prowadząc do prac lepszej jakości.
		
		%System nie informuje a~priori użytkownika o~zadaniu zbyt krótkiego czasu ruchu robota w~żadnym z~trybów pracy. 
		
		
		\section{Cel pracy}
		\label{Cel}
		Celem niniejszej pracy było stworzenie dodatkowej funkcjonalności systemu IrpOS, która nie pozwoliłaby użytkownikowi na zadanie za krótkiego czasu ruchu. Dodatkowo do terminala wysyłana jest informacja o~popełnionym błędzie. Skupiono się na stworzeniu rozwiązania działającego w~przestrzeni stawów. Nie zmieniono interfejsu użytkownika, a jedynie dodano analizę trajektorii po odebraniu polecania przez system. Obliczone zostają maksymalne prędkość i~przyśpieszenie potrzebne do osiągnięcia zadanej konfiguracji w~danym czasie dla kolejno każdego stawu. Jeśli którykolwiek ruch powoduje przekroczenie ograniczeń kinematycznych lub dynamicznych, wywołana zostaje wyżej wspomniana procedura. 
		
		\section{Struktura pracy}
		\label{Struktura}
		Praca składa się z~sześciu rozdziałów. Rozdział 2~przedstawia sprzęt i~narzędzia wykorzystywane do stworzenia pracy. Rozdział 3~prezentuje projekt rozwiązania wybranego problemu. Rozdział 4~opisuje jak realizowano zaproponowane rozwiązanie. Rozdział 5~skupia się na sposobie i~wynikach badania poprawności działania nowej funkcjonalności. Rozdział 6~podsumowuje wyniki pracy.

% TeX encoding = utf8
% TeX spellcheck = pl_PL 

\chapter{Wykorzystany sprzęt i~narzędzia}
\label{ch:sprzęt_i_narzędzia}

	\section{Robot IRp-6}
	\label{s:robot_irp6}
	Laboratorium 012 Wydziału Elektroniki i~Technik Informacyjnych wyposażone jest w~dwa roboty typu IRp-6. Bazują one na manipulatorach IRb-6, o pięciu stopniach swobody\cite{merapiap}, które odznaczyły się w~historii robotyki jako pierwsze elektryczne roboty przemysłowe sterowane za pomocą mikroprocesorów\cite{ABB}. Oba manipulatory Politechniki rozbudowano o~kiście pozwalające na obrót nadgarstka, a~jeden z~nich dodatkowo może poruszać się wzdłuż toru jezdnego. Ostatecznie manipulatory \textit{Postument} i~\textit{Track} posiadają kolejno sześć oraz siedem stopni swobody\cite{IRPOS}. Zastosowanie czujników położenia stawów pozwala na precyzyjny ruch, szczególnie ważny przy zadaniach przemysłowych i~badawczych. 
	\section{Robot Velma}
	\label{s:robot_velma}
	Robot Velma składa się z~czterech części:
	\begin{enumerate}
		\item obrotowego tułowia;
		\item ramion KUKA LWR;
		\item chwytaków BarrettHand;
		\item szyi, na której zamieszczona jest głowa z~kamerą Kinekt.
	\end{enumerate}
	Różnią się one między sobą stopni swobody oraz sposobem sterowania. Elementy nr 1 i~2 steruje się impedancyjnie, pozostałe pozycyjnie\cite{robotVelma}. Pełną strukturę sprzętu pokazuje rysunek~\ref{f:robot_Velma}.
	
	\begin{figure}[h]
		\centering
		\includegraphics[width=0.65\textwidth]{obrazy/velma_joints.png}
		\caption{Robot Velma (źródło: \cite{robotVelma})}
		\label{f:robot_Velma}
	\end{figure}
	
	Tak jak w~przypadku robotów Irp-6, szeroka gama czujników pozwala na dokładne sterowanie. Zaś sterowanie impedancyjne, a~konkretnie nadawanie pożądanej sztywności stawom, daje możliwość manipulacji dynamicznych relacji robota ze środowiskiem. Dodatkowo system Velmy wywołuje wirtualne siły odpychające od siebie stawy, tak by zapobiec ich kolizji. Zarówno obniżanie sztywności w~stawach, jak i~wprowadzanie wirtualnych sił zimniejsza prawdopodobieństwo idealnego osiągnięcia zadanej pozycji.
	\section{Oprogramowanie bazowe}
	\label{s:oprogramowanie_bazowe}
	Oprogramowanie, które steruje robotami Irp-6 i~Velma opiera się na wolnodostępnych platformach. Wprowadzone modyfikacje musiały zostać stworzone w~tej samej konwencji i~technologii. 
		\subsection{ROS}
		\label{ss:ros}
		ROS (The Robot Operating System) to otwarta platforma programistyczna, która przede wszystkim definiuje sposoby komunikacji między różnymi fragmentami systemu\cite{ROS}. Poza tym implementuje narzędzia przydatne nie tylko do czystej pracy systemu, ale i~jego analizy.
		
		ROS rozdziela elementy na węzły. Założeniem systemu jest dzielenie fragmentów oprogramowania na podstawie funkcjonalności. Przykładowo analizę położenia robota można rozdzielić na kolejne węzły: zbierający informację od enkoderów/kamer, konwertujący odczyty do bardziej czytelnej formy, określający położenie na podstawie dostosowanych danych. Widać tu potrzebę przesyłu danych pomiędzy wymienionymi węzłami. Istnieją dwa podstawowe sposoby komunikacji węzłów w~systemie opartym na tej platformie:
		\begin{itemize}
			\item za pomocą tematów - jeden z~węzłów pisze informacje na temacie (publisher), inne mogą je z~niego odczytywać (subscriber);
			\item komunikacja jeden-do-jednego - jeden z~węzłów zapewnia serwis, odpowiadający na zapytanie klienta.
		\end{itemize}
		Nie jest wykluczone by węzeł był jednocześnie odbiorcą i~dostarczycielem danych. Tak samo możliwe jest by komunikacja z~węzłem przebiegała na oba sposoby. Sposoby komunikacji przedstawiono jest na rysunku~\ref{f:ros_komunikacja}.
		\begin{figure}[h]
			\centering
			\includegraphics[width=0.8\textwidth]{obrazy/Concepts-de-base-de-ROS.jpg}
			\caption{Podstawowa komunikacja w~systemie ROS (źródło: \cite{whatROS})}
			\label{f:ros_komunikacja}
		\end{figure}
		
		
		Dodatkowo ROS wyposażony jest w bibliotekę \texttt{actionlib}. Jest to interfejs dla zadań, które można wywłaszczyć\cite{actionlibROS}. Jest to szczególnie przydatne gdy wykonanie czynności jest długotrwałe i~może zaburzyć działanie systemu. Wiadomości przekazywane pomiędzy klientem i~serwisem w~plikach deklaracji podzielone są na trzy części: gol, informacje zwrotne oraz wynik zadania. Serwer możne w~danym momencie mieć tylko jeden aktywny gol, o~którego statusie wysyłane są informacje zwrotne (np. w~każdej iteracji obliczającej ustawienia chwilowe). Biblioteka zapewnia wygody sposób na odwoływanie się do pól struktury wiadomości akcji oraz do operowania statusem celu (ustawianie go na przyjęty, odrzucony itp.). Ten sposób komunikacji jest wykorzystywany do łączenia interfejsu użytkownika sytemu IrpOS lub VelmOS z~komponentami akcji, a~przez to dedykowanymi im generatorami.
		
		\subsection{Open Robot Control Software}
		\label{ss:orocos}
		\subsection{Connman}
		\label{ss:connman}
		\subsection{Rviz}
		\label{ss:rviz}
		\subsection{Gazebo}
		\label{ss:gazebo}
	\section{System IrpOS}
	\label{s:irpos}
	\section{System VelmOS}
	\label{s:velmos}
		\subsection{rqt\_agent}
		\label{ss:rqt_agent}
		\subsection{show\_collisions}
		\label{ss:show_collisions}
		\subsection{show\_joints}
		\label{ss:show_joints}
		
	\section{Algorytmy}
	\label{s:algorytmy}

%\bibliographystyle{plain}
\bibliography{bibliografia}
%\listoffigures
%\listoftables
%\printglossary[title=Wykaz skrótów]

\end{document}
