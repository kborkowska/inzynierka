% TeX encoding = utf8
% !TeX spellcheck = pl_PL

\newpage
\thispagestyle{empty}
\fontsize{12}{12}\selectfont
\vspace*{2\baselineskip}
\begin{center}
	{\large\bfseries Streszczenie}\par\bigskip
\end{center}

\noindent{\bf Tytuł}: {Generacja trajektorii w systemie IRPOS}\par
\vspace*{0.5\baselineskip}

{\itshape
Celem pracy inżynierskiej była modyfikacja systemu IRPOS, umożliwiająca sprawdzenie, czy w~trakcie osiągania przez robota IRp6 punktów zadanej drogi nie zostaną naruszone jego ograniczenia fizyczne i~dynamiczne. A priori przeanalizowana zostaje trajektoria, którą poruszać się ma robot, by osiągnąć zadane położenie.
Zmodernizowany został komponent OROCOS, odpowiedzialny za odbieranie i~wstępne sprawdzenie zadania robota.
Dodatkowo nowa funkcjonalność wstrzymuje dalsze przetwarzanie rozkazu i~związany z~nim ruch. Jest to podejście zgodne z~wcześniej zaimplementowaną koncepcją reakcji na zadanie punktu końcowego niemożliwego do osiągnięcia dla danego stawu.
}

\vspace*{0.5\baselineskip}
\noindent{\bf Słowa kluczowe}: {\itshape robot IRp6, manipulator robotyczny, system IRPOS, planowanie trajektorii manipulatora, OROCOS, Open Robot Control Software.
	}

\par
\vspace{0\baselineskip}
\begin{center}
	{\large\bfseries Abstract}\par\bigskip
\end{center}

\noindent{\bf Title}: {\itshape Generation of robot trajectory in IRPOS system}\par
\vspace*{0.5\baselineskip}

{\itshape
	The aim of this thesis was modificating IRPOS system, that enables checking whether, during IRp6 robot's movement in the direction of setpoints, the physical and dynamic limitations of the robot are not breached. A priori a possible trajectory is analysed. OROCOS component, responsible for receiving and initial movement task test, was modernised.  Moreover, the new functionality blocks further setpoint message processing, thus halting robot's movement. This approach is in accordance with the previously implemented reaction to a setpoint that is impossible to reach by a given joint.
}

\vspace*{0.5\baselineskip}
\noindent{\bf Keywords}: {\itshape IRp6 robot, manipulator robot, IRPOS system, manipulator trajectory planning, OROCOS, Open Robot Control Software .}%, monocular odometry.}

\newpage
\thispagestyle{empty}
\mbox{}