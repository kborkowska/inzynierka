% TeX encoding = utf8
% TeX spellcheck = pl_PL 
\documentclass[a4paper, 12pt]{article}
\usepackage[utf8]{inputenc}
\usepackage[polish]{babel}
\usepackage{polski}
\usepackage{graphicx}
\usepackage{listings}
\usepackage{amsfonts}
\usepackage{geometry}
\usepackage{multicol}

\newgeometry{tmargin=3cm, bmargin=3cm, lmargin=2cm, rmargin=2cm}

\pagestyle{empty}

\title{Zapis zadań do pracy inżynierskiej \author{Karolina Borkowska}}

\begin{document}

	\maketitle

	\vspace{50px}

	W celu zaimplementowania algorytmu wykrywającego możliwe przekroczenie ograniczeń robota~w trakcie wykonywania zadanej trajektorii należy rozpoznać wyżej wymienione ograniczenia. Kolejne kroki zostaną wykonane by zidentyfikować właściwości robota:
	\begin{itemize}
	\item Implementacja komponentu OROCOS, który zastąpi obecny generator trajektorii na czas badań. Będzie on obliczać trapezoidalną trajektorię na podstawie podanych~w poleceniu wartości: punktu końcowego, przyśpieszenia oraz prędkości maksymalnej. 
	\item Zebranie danych o położeniu stawów do obliczenia uchybu. Przekroczenie wskaźnika jakości, bazującego na uchybie, wskazywać ma wykroczenie poza ograniczenia robota. Zapis i analiza danych powinna obywać się offline by nie wpływać na pracę regulatora.
	
	\end{itemize}
	
	
\end{document}


